\documentclass[journal,12pt,twocolumn]{IEEEtran}
%
\usepackage{setspace}
\usepackage{gensymb}
%\doublespacing
\singlespacing

%\usepackage{graphicx}
%\usepackage{amssymb}
%\usepackage{relsize}
\usepackage[cmex10]{amsmath}
%\usepackage{amsthm}
%\interdisplaylinepenalty=2500
%\savesymbol{iint}
%\usepackage{txfonts}
%\restoresymbol{TXF}{iint}
%\usepackage{wasysym}
\usepackage{amsthm}
%\usepackage{iithtlc}
\usepackage{mathrsfs}
\usepackage{txfonts}
\usepackage{stfloats}
\usepackage{bm}
\usepackage{cite}
\usepackage{cases}
\usepackage{subfig}
%\usepackage{xtab}
\usepackage{longtable}
\usepackage{multirow}
%\usepackage{algorithm}
%\usepackage{algpseudocode}
\usepackage{enumitem}
\usepackage{mathtools}
\usepackage{steinmetz}
\usepackage{tikz}
\usepackage{circuitikz}
\usepackage{verbatim}
\usepackage{tfrupee}
\usepackage[breaklinks=true]{hyperref}
%\usepackage{stmaryrd}
\usepackage{tkz-euclide} % loads  TikZ and tkz-base
%\usetkzobj{all}
\usetikzlibrary{calc,math}
\usepackage{listings}
    \usepackage{color}                                            %%
    \usepackage{array}                                            %%
    \usepackage{longtable}                                        %%
    \usepackage{calc}                                             %%
    \usepackage{multirow}                                         %%
    \usepackage{hhline}                                           %%
    \usepackage{ifthen}                                           %%
  %optionally (for landscape tables embedded in another document): %%
    \usepackage{lscape}     
\usepackage{multicol}
\usepackage{chngcntr}
%\usepackage{enumerate}

%\usepackage{wasysym}
%\newcounter{MYtempeqncnt}
\DeclareMathOperator*{\Res}{Res}
%\renewcommand{\baselinestretch}{2}
\renewcommand\thesection{\arabic{section}}
\renewcommand\thesubsection{\thesection.\arabic{subsection}}
\renewcommand\thesubsubsection{\thesubsection.\arabic{subsubsection}}

\renewcommand\thesectiondis{\arabic{section}}
\renewcommand\thesubsectiondis{\thesectiondis.\arabic{subsection}}
\renewcommand\thesubsubsectiondis{\thesubsectiondis.\arabic{subsubsection}}

% correct bad hyphenation here
\hyphenation{op-tical net-works semi-conduc-tor}
\def\inputGnumericTable{}                                 %%

\lstset{
%language=C,
frame=single, 
breaklines=true,
columns=fullflexible
}
%\lstset{
%language=tex,
%frame=single, 
%breaklines=true
%}

\begin{document}
%


\newtheorem{theorem}{Theorem}[section]
\newtheorem{problem}{Problem}
\newtheorem{proposition}{Proposition}[section]
\newtheorem{lemma}{Lemma}[section]
\newtheorem{corollary}[theorem]{Corollary}
\newtheorem{example}{Example}[section]
\newtheorem{definition}[problem]{Definition}
%\newtheorem{thm}{Theorem}[section] 
%\newtheorem{defn}[thm]{Definition}
%\newtheorem{algorithm}{Algorithm}[section]
%\newtheorem{cor}{Corollary}
\newcommand{\BEQA}{\begin{eqnarray}}
\newcommand{\EEQA}{\end{eqnarray}}
\newcommand{\define}{\stackrel{\triangle}{=}}

\bibliographystyle{IEEEtran}
%\bibliographystyle{ieeetr}


\providecommand{\mbf}{\mathbf}
\providecommand{\pr}[1]{\ensuremath{\Pr\left(#1\right)}}
\providecommand{\qfunc}[1]{\ensuremath{Q\left(#1\right)}}
\providecommand{\sbrak}[1]{\ensuremath{{}\left[#1\right]}}
\providecommand{\lsbrak}[1]{\ensuremath{{}\left[#1\right.}}
\providecommand{\rsbrak}[1]{\ensuremath{{}\left.#1\right]}}
\providecommand{\brak}[1]{\ensuremath{\left(#1\right)}}
\providecommand{\lbrak}[1]{\ensuremath{\left(#1\right.}}
\providecommand{\rbrak}[1]{\ensuremath{\left.#1\right)}}
\providecommand{\cbrak}[1]{\ensuremath{\left\{#1\right\}}}
\providecommand{\lcbrak}[1]{\ensuremath{\left\{#1\right.}}
\providecommand{\rcbrak}[1]{\ensuremath{\left.#1\right\}}}
\theoremstyle{remark}
\newtheorem{rem}{Remark}
\newcommand{\sgn}{\mathop{\mathrm{sgn}}}
\providecommand{\abs}[1]{\left\vert#1\right\vert}
\providecommand{\res}[1]{\Res\displaylimits_{#1}} 
\providecommand{\norm}[1]{\left\lVert#1\right\rVert}
%\providecommand{\norm}[1]{\lVert#1\rVert}
\providecommand{\mtx}[1]{\mathbf{#1}}
\providecommand{\mean}[1]{E\left[ #1 \right]}
\providecommand{\fourier}{\overset{\mathcal{F}}{ \rightleftharpoons}}
%\providecommand{\hilbert}{\overset{\mathcal{H}}{ \rightleftharpoons}}
\providecommand{\system}{\overset{\mathcal{H}}{ \longleftrightarrow}}
	%\newcommand{\solution}[2]{\textbf{Solution:}{#1}}
\newcommand{\solution}{\noindent \textbf{Solution: }}
\newcommand{\cosec}{\,\text{cosec}\,}
\providecommand{\dec}[2]{\ensuremath{\overset{#1}{\underset{#2}{\gtrless}}}}
\newcommand{\myvec}[1]{\ensuremath{\begin{pmatrix}#1\end{pmatrix}}}
\newcommand{\mydet}[1]{\ensuremath{\begin{vmatrix}#1\end{vmatrix}}}
%\numberwithin{equation}{section}
\numberwithin{equation}{subsection}
%\numberwithin{problem}{section}
%\numberwithin{definition}{section}
\makeatletter
\@addtoreset{figure}{problem}
\makeatother

\let\StandardTheFigure\thefigure
\let\vec\mathbf
%\renewcommand{\thefigure}{\theproblem.\arabic{figure}}
\renewcommand{\thefigure}{\theproblem}
%\setlist[enumerate,1]{before=\renewcommand\theequation{\theenumi.\arabic{equation}}
%\counterwithin{equation}{enumi}


%\renewcommand{\theequation}{\arabic{subsection}.\arabic{equation}}

\def\putbox#1#2#3{\makebox[0in][l]{\makebox[#1][l]{}\raisebox{\baselineskip}[0in][0in]{\raisebox{#2}[0in][0in]{#3}}}}
     \def\rightbox#1{\makebox[0in][r]{#1}}
     \def\centbox#1{\makebox[0in]{#1}}
     \def\topbox#1{\raisebox{-\baselineskip}[0in][0in]{#1}}
     \def\midbox#1{\raisebox{-0.5\baselineskip}[0in][0in]{#1}}

\vspace{3cm}


\title{Assignment 16}
\author{Jayati Dutta}





% make the title area
\maketitle

\newpage

%\tableofcontents

\bigskip

\renewcommand{\thefigure}{\theenumi}
\renewcommand{\thetable}{\theenumi}
%\renewcommand{\theequation}{\theenumi}


\begin{abstract}
This is a simple document explaining how to determine whether a set of polynomials are linearly independent or not.
\end{abstract}

%Download all python codes 
%
%\begin{lstlisting}
%svn co https://github.com/JayatiD93/trunk/My_solution_design/codes
%\end{lstlisting}

Download all and latex-tikz codes from 
%
\begin{lstlisting}
svn co https://github.com/gadepall/school/trunk/ncert/geometry/figs
\end{lstlisting}
%


\section{Problem}
Let $\vec{S}$ be a set of non-zero polynomials over a field $F$. If no two elements of $\vec{S}$ have the same degree, show that $\vec{S}$ is an independent set in $\vec{F[x]}$.
\section{Solution}
\begin{table}[h!]
\begin{center}
\begin{tabular}{|c|c|}
\hline
& \\
Given & $\vec{S}$ be a set of non-zero\\
& polynomial over a field $F$\\
& \\
& No two elements of $\vec{S}$ have\\
& the same degree\\
& \\
\hline
& \\
To prove & $\vec{S}$ is an independent set\\
& in $\vec{F[x]}$\\
& \\
\hline
& \\
Linear &\\
Independency  & Let $f_1$,$f_2$,...,$f_n$ are the\\
& polynomials and they will be \\
& linearly independent if\\
& $a_1f_1+a_2f_2+...+a_nf_n=\theta$ \\
& for $a_1=a_2=....=a_n=0$ \\
& where $a_1$,$a_2$,...,$a_n$ are \\
& scalars from field $F$\\
\hline
\end{tabular}
\end{center}
\end{table}


\begin{table}[h!]
\begin{center}
\begin{tabular}{|c|c|}
\hline
& \\
Proof & Let the degrees of $f_1$,$f_2$,...,$f_n$ are\\
& $d_1$,$d_2$,...,$d_n$ respectively such that\\
& the degree of $f_i$ = $d_i \neq d_j$\\
& for j=1,2,....,n and $i \neq j$\\
& and $d_1 < d_2 < ....< d_n$\\
& so $d_n$ is the largest degree\\
& \\
& Now, let $a_1f_1+a_2f_2+...+a_nf_n=\theta$\\
& \\
& where $f_1 = \sum_{i=0}^{d_1}k_{1i}x^i$\\
& $f_2 = \sum_{i=0}^{d_2}k_{2i}x^i$\\
& $\textbf{or,}f_2 = \sum_{i=0}^{d_1}k_{2i}x^i+ \sum_{i=d_1+1}^{d_2}k_{2i}x^i$\\
& \\
& Similarly, $f_{n-1}=\sum_{i=0}^{d_{n-1}}k_{(n-1)i}x^i$\\
& $\implies f_{n-1}= \sum_{i=0}^{d_{n-2}}k_{(n-1)i}x^i+ \sum_{i=d_{n-2}+1}^{d_{n-1}}k_{(n-1)i}x^i$\\
& and $f_n = \sum_{i=0}^{d_{n-1}}k_{ni}x^i+ \sum_{i=d_{n-1}+1}^{d_{n}}k_{ni}x^i$\\
& \\
& Now,\\
& $a_1f_1+a_2f_2+...+a_nf_n$\\
& $=a_1\sum_{i=0}^{d_1}k_{1i}x^i + a_2(\sum_{i=0}^{d_1}k_{2i}x^i+ \sum_{i=d_1+1}^{d_2}k_{2i}x^i)$\\
& $+ ...+a_n(\sum_{i=0}^{d_1}k_{ni}x^i+ \sum_{i=d_1+1}^{d_2}k_{ni}x^i+..$\\
& $..+\sum_{i=d_{n-1}+1}^{d_{n}}k_{ni}x^i)$\\
& $=\sum_{i=0}^{d_1}(a_1k_{1i}+a_2k_{2i}+..+a_nk_{ni})x^i$\\
& $+\sum_{i=d_1+1}^{d_2}(a_2k_{2i}+..+a_nk_{ni})x^i$\\
& $+..+\sum_{i=d_{n-1}+1}^{d_{n}}a_nk_{ni}x^i$\\
& \\
& Now, as $a_1f_1+a_2f_2+...+a_nf_n=\theta$ \\
& for $d_{n-1}+1\leq i \leq d_n$, $k_{ni}\neq 0$\\
& so $a_n$ must be 0\\
& \\
& Now, discarding $a_n$ associated term, we get\\
& $\sum_{i=0}^{d_1}(a_1k_{1i}+a_2k_{2i}+..+a_nk_{ni})x^i$\\
& $+\sum_{i=d_1+1}^{d_2}(a_2k_{2i}+..+a_nk_{ni})x^i$\\
& $+..+\sum_{i=d_{n-2}+1}^{d_{n-1}}a_{n-1}k_{(n-1)i}x^i =0$\\
& so, for $d_{n-2}+1\leq i \leq d_{n-1}$, $k_{(n-1)i}\neq 0$\\
& $\implies a_{n-1}=0$ \\
%& In this way, it can be proved that\\
%& $a_1=a_2=....=a_n=0$ for\\
%& $a_1f_1+a_2f_2+...+a_nf_n=\theta$\\
%& $\implies$ $f_1$,$f_2$,....,$f_n$ are\\
%& linearly independent\\
& \\
\hline
\end{tabular}
\end{center}
\end{table}

\begin{table}[h!]
\begin{center}
\begin{tabular}{|c|c|}
\hline
& \\
Proof & \\
& Similarly, for $d_1+1 \leq i \leq d_2$\\
& $\sum_{i=0}^{d_1}(a_1k_{1i}+a_2k_{2i})x^i + \sum_{i=d_1+1}^{d_2}a_2k_{2i}x^i$\\
& $\implies a_2=0$ as $k_{2i} \neq 0$\\
& $\implies \sum_{i=0}^{d_1}a_1k_{1i}x^i=0$\\
& $\implies a_1=0$ as $k_{1i} \neq 0$\\
& In this way, it can be proved that\\
& $a_1=a_2=....=a_{n-1}=a_n=0$ for\\
& $a_1f_1+a_2f_2+...+a_nf_n=\theta$\\
& $\implies$ $f_1$,$f_2$,....,$f_n$ are\\
& linearly independent\\
& \\
\hline
\end{tabular}
\end{center}
\end{table}


Hence, it is proved that $\vec{S}$ is an independent set in $\vec{F[x]}$.
%\renewcommand{\theequation}{\theenumi}
%\begin{enumerate}[label=\thesection.\arabic*.,ref=\thesection.\theenumi]
%\numberwithin{equation}{enumi}
%\item Verification of the above problem using python code.\\
%\solution The  following Python code verifies the above solution.
%\begin{lstlisting}
%codes/multiplication_test.py
%\end{lstlisting}
%%%
%\end{enumerate}

\end{document}



